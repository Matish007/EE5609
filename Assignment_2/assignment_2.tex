\documentclass[journal,12pt,twocolumn]{IEEEtran}

\usepackage{setspace}
\usepackage{gensymb}


\singlespacing

\usepackage[cmex10]{amsmath}
%\usepackage{amsthm}
%\interdisplaylinepenalty=2500
%\savesymbol{iint}
%\usepackage{txfonts}
%\restoresymbol{TXF}{iint}
%\usepackage{wasysym}
\usepackage{amsthm}

\usepackage{mathrsfs}
\usepackage{txfonts}
\usepackage{stfloats}
\usepackage{bm}
\usepackage{cite}
\usepackage{cases}
\usepackage{subfig}

\usepackage{longtable}
\usepackage{multirow}

\usepackage{enumitem}
\usepackage{mathtools}
\usepackage{steinmetz}
\usepackage{tikz}
\usepackage{circuitikz}
\usepackage{verbatim}
\usepackage{tfrupee}
\usepackage[breaklinks=true]{hyperref}

\usepackage{tkz-euclide} %loads TikZ and tkz-base

\usetikzlibrary{calc,math}
\usepackage{listings}
    \usepackage{color}                                          
    \usepackage{array}                                          
    \usepackage{longtable}                                      
    \usepackage{calc}                                           
    \usepackage{multirow}                                       
    \usepackage{hhline}                                         
    \usepackage{ifthen}
    \usepackage{lscape}     
\usepackage{multicol}
\usepackage{chngcntr}

\DeclareMathOperator*{\Res}{Res}

\renewcommand\thesection{\arabic{section}}
\renewcommand\thesubsection{\thesection.\arabic{subsection}}
\renewcommand\thesubsubsection{\thesubsection.\arabic{subsubsection}}

\renewcommand\thesectiondis{\arabic{section}}
\renewcommand\thesubsectiondis{\thesectiondis.\arabic{subsection}}
\renewcommand\thesubsubsectiondis{\thesubsectiondis.\arabic{subsubsection}}

\hyphenation{op-tical net-works semi-conduc-tor}
\def\inputGnumericTable{}                                 %%

\lstset{
%language=C,
frame=single, 
breaklines=true,
columns=fullflexible
}

\begin{document}

\newtheorem{theorem}{Theorem}[section]
\newtheorem{problem}{Problem}
\newtheorem{proposition}{Proposition}[section]
\newtheorem{lemma}{Lemma}[section]
\newtheorem{corollary}[theorem]{Corollary}
\newtheorem{example}{Example}[section]
\newtheorem{definition}[problem]{Definition}

\newcommand{\BEQA}{\begin{eqnarray}}
\newcommand{\EEQA}{\end{eqnarray}}
\newcommand{\define}{\stackrel{\triangle}{=}}
\bibliographystyle{IEEEtran}
\providecommand{\mbf}{\mathbf}
\providecommand{\pr}[1]{\ensuremath{\Pr\left(#1\right)}}
\providecommand{\qfunc}[1]{\ensuremath{Q\left(#1\right)}}
\providecommand{\sbrak}[1]{\ensuremath{{}\left[#1\right]}}
\providecommand{\lsbrak}[1]{\ensuremath{{}\left[#1\right.}}
\providecommand{\rsbrak}[1]{\ensuremath{{}\left.#1\right]}}
\providecommand{\brak}[1]{\ensuremath{\left(#1\right)}}
\providecommand{\lbrak}[1]{\ensuremath{\left(#1\right.}}
\providecommand{\rbrak}[1]{\ensuremath{\left.#1\right)}}
\providecommand{\cbrak}[1]{\ensuremath{\left\{#1\right\}}}
\providecommand{\lcbrak}[1]{\ensuremath{\left\{#1\right.}}
\providecommand{\rcbrak}[1]{\ensuremath{\left.#1\right\}}}
\theoremstyle{remark}
\newtheorem{rem}{Remark}
\newcommand{\sgn}{\mathop{\mathrm{sgn}}}
\providecommand{\abs}[1]{\left\vert#1\right\vert}
\providecommand{\res}[1]{\Res\displaylimits_{#1}} 
\providecommand{\norm}[1]{\left\lVert#1\right\rVert}
%\providecommand{\norm}[1]{\lVert#1\rVert}
\providecommand{\mtx}[1]{\mathbf{#1}}
\providecommand{\mean}[1]{E\left[ #1 \right]}
\providecommand{\fourier}{\overset{\mathcal{F}}{ \rightleftharpoons}}
%\providecommand{\hilbert}{\overset{\mathcal{H}}{ \rightleftharpoons}}
\providecommand{\system}{\overset{\mathcal{H}}{ \longleftrightarrow}}
	%\newcommand{\solution}[2]{\textbf{Solution:}{#1}}
\newcommand{\solution}{\noindent \textbf{Solution: }}
\newcommand{\cosec}{\,\text{cosec}\,}
\providecommand{\dec}[2]{\ensuremath{\overset{#1}{\underset{#2}{\gtrless}}}}
\newcommand{\myvec}[1]{\ensuremath{\begin{pmatrix}#1\end{pmatrix}}}
\newcommand{\mydet}[1]{\ensuremath{\begin{vmatrix}#1\end{vmatrix}}}
\numberwithin{equation}{subsection}
\makeatletter
\@addtoreset{figure}{problem}
\makeatother
\let\StandardTheFigure\thefigure
\let\vec\mathbf
\renewcommand{\thefigure}{\theproblem}
\def\putbox#1#2#3{\makebox[0in][l]{\makebox[#1][l]{}\raisebox{\baselineskip}[0in][0in]{\raisebox{#2}[0in][0in]{#3}}}}
    \def\rightbox#1{\makebox[0in][r]{#1}}
 \def\centbox#1{\makebox[0in]{#1}}
     \def\topbox#1{\raisebox{-\baselineskip}[0in][0in]{#1}}
     \def\midbox#1{\raisebox{-0.5\baselineskip}[0in][0in]{#1}}
\vspace{3cm}
\title{EE5609 Assignment 2}
\author{Gaydhane Vaibhav Digraj \\ RollNo : AI20MTECH11002 }
\maketitle
\newpage
%\tableofcontents
\bigskip
\renewcommand{\thefigure}{\theenumi}
\renewcommand{\thetable}{\theenumi}
\begin{abstract}
This assignment involves finding the matrix \vec{X} by solving the equation. 
\end{abstract}
The python code solution can be downloaded from
\begin{lstlisting}
https://github.com/Vaibhav11002/EE5609/blob/master/Assignment_2/Codes/assignment_2.py
\end{lstlisting}

\section{Problem}
Find $\vec{X}$ if $\vec{Y} = \myvec{ 3 & 2 \\ 1 & 4 }$ and $2\vec{X}+\vec{Y} = \myvec{1 & 0 \\ -3 & 2}$. Express $2\vec{X}+\vec{Y} = \vec{A}\vec{B}$, where B is a block matrix comprising of $\vec{X}$ and $\vec{Y}$ and find the matrix $\vec{A}$. 
\vspace{2mm}

\section{Solution}
We have, 
\begin{align}
2\vec{X}+\vec{Y} &= \myvec{1 & 0 \\ -3 & 2 } \\
\implies 2\vec{X} &= \myvec{1 & 0 \\ -3 & 2 } - \vec{Y} \\
2\vec{X} &= \myvec{1 & 0 \\ -3 & 2 } - \myvec{3 & 2 \\ 1 & 4} \notag \\
&= \myvec{-2 & -2 \\ -4 & -2 }
\end{align}

Now, 
\begin{align}
\vec{X} &= \frac{1}{2}\myvec{-2 & -2 \\ -4 & -2 } \\
&= \myvec{-2/2 & -2/2 \\ -4/2 & -2/2 }\notag \\
&= \myvec{-1 & -1 \\ -2 & -1}
\end{align}

Thus from (2.0.5) we get, 

\begin{align*}
    \boxed{ \vec{X} = \myvec{-1 & -1 \\ -2 & -1} }
\end{align*}

From (2.0.1), 
\begin{align}
    2\vec{X}+\vec{Y} &=\vec{A}\vec{B} \\
    &= \myvec{1 & 0 \\ -3 & 2}
\end{align}
Where B is a block matrix comprising $\vec{X}$ and $\vec{Y}$. So, 
\begin{align}
    \vec{B} &= \myvec{\vec{X} \\ \vec{Y}}
\end{align}
Now, 
\begin{align}
    \vec{A}\vec{B} &= \vec{A}\myvec{\vec{X} \\ \vec{Y}} \notag\\
    &= \myvec{1 & 0 \\ -3 & 2} 
\end{align}

Since $\vec{B}$ is a 4x2 matrix, $\vec{A}$ should be 2x4 so that the product $\vec{A}\vec{B}$ is a 2x2 matrix.\\ 
Let $\vec{A}$ be a block matrix comprising of $\vec{A_{1}}$ and
$\vec{A_{2}}$. 
\begin{align}
    \vec{A} &= \myvec{\vec{A_{1}}& \vec{A_{2}}}
\end{align}
Now, 
\begin{align}
    \vec{A}\vec{B} &= \myvec{\vec{A_{1}}& \vec{A_{2}}}\myvec{\vec{X} \\ \vec{Y}}\notag\\
    &= \vec{A_{1}}\vec{X} + \vec{A_{2}}\vec{Y}
\end{align}
From (2.0.6) we get, 
\begin{align}
    2\vec{X}+\vec{Y} &= (2\vec{I})\vec{X} + \vec{I}\vec{Y}\notag\\
    &=\vec{A_{1}}\vec{X} + \vec{A_{2}}\vec{Y}
\end{align}
Thus, $\vec{A_{1}} = 2\vec{I}$ and $\vec{A_{2}} = \vec{I}$ \\
Hence we get the matrix, 
\begin{align*}
    \boxed{\vec{A} =  \myvec{2 & 0 & 1 & 0 \\ 0 & 2 & 0 & 1}}
\end{align*}
\end{document}
